\documentclass[10pt]{article}
\usepackage{NotesTeX,lipsum}

\title{\begin{center}{\Huge paper reading notes of}\\{{\itshape nonclassical photon sources}}\end{center}}
\author{Loeng Waidak}

\affiliation{
    Sun-Yatsen unniversity\\
    School of Physics and Astronomy
}

\emailAdd{liangwd23@mail2.sysu.edu.cn}

\newcommand{\h}[1]{\hat{#1}}
\newcommand{\hd}[1]{\hat{#1}^{\dagger}}
\renewcommand{\bra}[1]{\langle #1 |}
\renewcommand{\ket}[1]{| #1 \rangle}
\begin{document}
    \maketitle
    \flushbottom
    \newpage
    \pagestyle{fancynotes}

    \part{Basic Concept}
    \section{Jaynes-Cumings model}
        Jaynes-Cumings(JC) model is used to describe the coupling between atom and light within the cavity.
        It use quantum theory to describe both light and atom.
        The simplest JC model's Hamiltonian can be write as:
        \begin{equation}
            \h{H} = \hbar \omega \hd{a} \h{a} + \frac{1}{2}\hbar \omega_A \h{\sigma}_z - \hbar g (\h{a}\hd{\sigma} + \hd{a}\h{\sigma})
        \end{equation}
        It is obviously that it couples atom state $\ket{g}$ and $\ket{e}$ with number state of light $\ket{n}$.
        We can easily calculate martix elements like:
        \begin{equation}
            \bra{n+1,g} \h{H} \ket{n+1,g} = (n+1)\omega - \frac{1}{2}\omega_A
        \end{equation}
        Therefore we can rewrite JC Hamiltonian into $2\times 2$ matrix as:
        \begin{equation}
            \h{H} \begin{pmatrix}
                \ket{n+1,g} \\ \ket{n,e}
            \end{pmatrix}
            = \hbar \begin{pmatrix}
                (n+1)\omega-\frac{1}{2}\omega_A & -g\sqrt{n+1}\\
                -g\sqrt{n+1} & n\omega + \frac{1}{2}\omega_A
            \end{pmatrix}
            \begin{pmatrix}
                \ket{n+1,g} \\ \ket{n,e}
            \end{pmatrix}
        \end{equation}

        
        \begin{proof}
            Now we start to solve eigenvalue problem!
            \begin{align}
                \left[(n+1)\omega - \frac{1}{2}\omega_A - \lambda\right]\left[n\omega + \frac{1}{2}\omega_A - \lambda\right]-g^2(n+1) &=0 \\
                \lambda^2 - (2n+1)\omega \lambda + (n+1)^2\omega^2 - \frac{\delta^2}{4} - \frac{\Omega^2_n}{4} &= 0 
            \end{align}
            Easily to derive that:
            \begin{equation}
                \Delta = \delta^2 + \Omega^2_n 
            \end{equation}
            \begin{align}
                \lambda &= \frac{(2n+1)\omega \pm \sqrt{\Delta}}{2} \\
                &=(n+\frac{1}{2})\omega \pm\frac{1}{2}\sqrt{\delta^2 + \Omega_n^2} \\
                &= (n+\frac{1}{2})\omega \pm \frac{1}{2} \delta^2 + \Omega^2_n 
            \end{align}
            where
            \begin{equation}
                \delta = \omega - \omega_A \qquad \Omega_n = 2g\sqrt{n+1}
            \end{equation}
        \end{proof}
        This mean the state $\ket{n+1,g}$ and $\ket{n,e}$ are couple and struct two new state, whose eigenvalues is $\lambda_{\pm} = (n+\frac{1}{2})\omega \pm \frac{1}{2} \Delta_n$ due to the interactions between light field and atom.
        And now we can findout the eigenstate of this coupling system.
        \begin{proof}
            \begin{equation}
                \h{H}_n \ket{n,\pm} = \lambda_{\pm} \ket{n,\pm}
            \end{equation}
            we can let $\ket{n,+}$ write as:
            \begin{equation}
                \ket{n,+} = 
                \begin{pmatrix}
                    a\\b
                \end{pmatrix}
            \end{equation}
            then we need to solve the equation to find a and b :
            \begin{align}
                \left[(n+1)\omega - \frac{1}{2}\omega_A \right]a - (g\sqrt{n+1}) b &= \left[(n+\frac{1}{2})\omega + \frac{1}{2}\sqrt{\delta^2 + \Omega_n^2}\right]a\\
                (\omega - \omega_A + \sqrt{\delta^2 + \Omega^2_n}) a&= (2g\sqrt{n+1})b\\
                (\delta - \Delta_n)a &= \Omega_n b
            \end{align}
            where
            \begin{equation}
                \Delta_n = \sqrt{\delta^2 + \Omega^2}
            \end{equation}
            the possible solution of a,b is :
            \begin{gather}
                a = \frac{\Omega_n}{\sqrt{(\delta-\Delta_n)^2 + \Omega_n}}\\
                b = \frac{\delta - \Delta_n}{\sqrt{(\delta-\Delta_n)^2 + \Omega_n}}
            \end{gather}
            Therefore
            \begin{equation}
                \ket{n,+} = \frac{\Omega_n}{\sqrt{(\delta-\Delta_n)^2 + \Omega_n}}\ket{n+1,g} + \frac{\delta - \Delta_n}{\sqrt{(\delta-\Delta_n)^2 + \Omega_n}} \ket{n,e}
            \end{equation}
            In the same way, we can attain :
            \begin{equation}
                \ket{n,-} = \frac{\Delta_n - \delta}{\sqrt{(\delta-\Delta_n)^2 + \Omega_n}} \ket{n+1,g} + \frac{\Omega_n}{\sqrt{(\delta-\Delta_n)^2 + \Omega_n}} \ket{n,e}
            \end{equation}
            for convenience, we better define :
            \begin{equation}
                \cos\Theta = \frac{\Omega_n}{\sqrt{(\delta-\Delta_n)^2 + \Omega_n}}, \qquad \sin\Theta = \frac{\Delta_n - \delta}{\sqrt{(\delta-\Delta_n)^2 + \Omega_n}}
            \end{equation}
            then we can simplify eigenstate as\mn{This solution are different from the textbook of JC model. It may has some wrong with that textbook.} :
            \begin{align}
                \ket{n,+} &= \cos\Theta \ket{n+1,g} - \sin\Theta \ket{n,e}\\
                \ket{n,-} &= \sin\Theta \ket{n+1,g} + \sin\Theta \ket{n,e}
            \end{align}
        \end{proof}

        \section{Correlation function}
        \section{Master equation}

    \part{Research Background}
        \section{Original motivation of this field}
            In the applications of quantum infomation science such as: quantum computating, cryptography and metrology, generate and manipulate
            single photon is an necessary technic. The biggest motivation to implement high quailty single photon source is that the born of quantum
            computer will bring an extordinary impact on all over the world.
        \section{Important achievements}
            Up to now, several system including atom-cavity coupled with micro-cavity system, single quantum dot integrated with photonic crystal cavity, 
            optical fibers and surface plasmos had been demonstrated that it can observe and manipulate single photon.
        \section{Applications potential of this field}

    \part{Study of scheme}
        \section{Atom-Cavity coupled}
            Dterministic single-photon sources have been realized with newutral atoms, embedded molecules, trapped ions, quantum dots and defect centres.
            But different apolications have different requirement. For example, quantum computing or quantum networking require that photons must also be indistinguishable
            and high efficiency. Unfortunately, high efficiency is hard to obtain in free space where atoms only contain between two lens. But strongly coupling 
            the radiating object to an optical microcavity can raise up the efficiency.\par

            Compare to ions, atoms are largely immune to perturbations like electric patch fields close to dielectric mirrors.

            Markus \textit{et al}. report a scheme of Rb atom which can save Rb atoms in the cavity up to 30s for producing up to 300,000 photons per atom.\cite{hijlkema_single-photon_2007}
        \section{Quantum dot - Cavity coupled}
        Due to their corresponding atomic-like discrete energy levels, semiconductor quantum dots are often given the moniker artificial atoms. And thanks
        to its poperty of designable, quantum dots based single photon source has the band from IR at $1.55 \mu m$ to the deep UV region at $<280 nm$.
        More than this, single photon emission from semiconductor QDs has also been realized at temperatures up to room temperature and beyond.

            \subsection{In the Original region ~ 1300nm}
                Miyazawa \textit{et al.} use InAs QDs obtain lowest $g^2(0) = (4.4\pm0.2)\times 10^{-4}$.\cite{miyazawa_single-photon_2016} About 200KM QKD require $g^2(0)$ down to the 
                value $10^{-4}$ and the parameters $\langle n \rangle$ and the radiative lifetime must be improved to 0.175 and 0.12 ns, respectively.
                Miyazawa \textit{et al.} already achieve one of the requirements.

            \subsection{The visible}
                Quantum dots formed from InGaN, InP, and various combinations of type II–VI semiconductors such as 
                CdSe/ZnSe can be used to generate single photons with wavelengths in the visible region of the spectrum.
                Moreover, in contrast to the long wavelength emitters discussed above, single photon emission in the visible 
                and ultraviolet has regularly been reported from QDs at ambient temperatures of 300 K and above.\par

                Fedorych \textit{et al.} realized room temperature single photon emission from CdSe/ZnSSe quantum dots. They reduce the correlation function
                $g^{(2)}(0)$ down to $0.16\pm0.15$ at $T=300K$\cite{fedorych_room_2012}.\par

                \begin{definition}
                    [Advantage] Can be obtain under romm temperature, which is an key point to ahcieve widly applications.
                    [Drawback] Due to high temperature, the efficiency is relatively low.
                \end{definition}

            \subsection{The UV region}
                Tamariz \textit{et al.} demonstrated GaN QD single photon emitter can be operated at 300K with $g^{(2)(0)= 0.17\pm 0.08}$. In addition,
                photon emission rates can raise up to $6\times 10^{6} s^{-1}$ while $g^{(2)}(0)\leq 0.5$ is maintained\cite{tamariz_towards_2020}.
        \section{Optical fibers}
        \section{Surface plasmos}


    \bibliographystyle{elsarticle-num}
    \bibliography{sps}
\end{document}